%% SECTION 3.1 %%
\section{Azuma-Hoeffding Inequality}

To state the main result of this section, we first need to introduce the concept of \emph{martingales}.

\begin{definition}
Let $(\Omega, \mathcal{A}, \mathbb{P})$ be a probability space.
\begin{enumerate}[(i)]
    \item A \emph{filtration} $\set{\mathcal{F}_n}_{n \in \N}$ is an increasing sequence of sub-$\sigma$-algebras $\mathcal{F}_n$ of $\mathcal{A}$, i.e.,
    \[
        \mathcal{F}_1 \subseteq \mathcal{F}_2 \subseteq \dots \subseteq \mathcal{A}.
    \]

    \item A stochastic process $\set{X_n}_{n \in \N}$ is called a \emph{martingale} if, for every $n \in \N$,
    \begin{enumerate}[(a)]
        \item $X_n$ is $\mathcal{F}_n$-measurable and integrable, i.e., $\Ex{\abs{X_n}} < \infty$,

        \item $\Ex{X_{n+1} \given \mathcal{F}_n} = X_n$ holds almost surely.
    \end{enumerate}
\end{enumerate}
\end{definition}

Loosely speaking, martingales are a generalization of sums of zero-mean, independent random variables: Let $\set{X_n}_{n \in \N}$ be a sequence of independent random variables with $\Ex{X_n} = 0$, and let $S_n = \sum_{i=1}^n X_i$. Then, for $\mathcal{F}_n = \sigma(X_1, \dots, X_n)$, we have
\[
    \Ex{S_{n+1} \given \mathcal{F}_n} = \Ex{X_{n+1} + S_n \given \mathcal{F}_n} = \Ex{X_{n+1}} + \Ex{S_n \given \mathcal{F}_n} = S_n,
\]
since $X_{n+1}$ is centered and independent of $\mathcal{F}_n$, and $S_n$ is $\mathcal{F}_n$-measurable.

Another definition we will need is that of a \emph{martingale difference sequence}.

\begin{definition}
Let $(\Omega, \mathcal{A}, \mathbb{P})$ be a probability space with filtration $\set{\mathcal{F}_n}_{n \in \N}$. A stochastic process $\set{X_n}_{n\in\N}$ is called a \emph{martingale difference sequence (MDS)}, if
\begin{enumerate}[(i)]
    \item $X_n$ is $\mathcal{F}_n$-measurable and integrable, i.e., $\Ex{\abs{X_n}} < \infty$,

    \item $\Ex{X_{n+1} \given \mathcal{F}_n} = 0$ holds almost surely.
\end{enumerate}
\end{definition}

Whenever we have a martingale $\set{X_n}_{n\in\N}$, we can construct a martingale difference sequence by setting $\Delta_n = X_n - X_{n-1}$, since then
\[
    \Ex{\Delta_{n+1} \given \mathcal{F}_n} = \Ex{X_{n+1} - X_n \given \mathcal{F}_n} = \Ex{X_{n+1} \given \mathcal{F}_n} - X_n = 0.
\]

\begin{theorem}[Azuma-Hoeffding]
\label{thm: azuma-hoeffding}
Assume that $\set{\Delta_i}_{i\in\N}$ is a martingale difference sequence with respect to a filtration $\set{\mathcal{F}_i}_{i\in\N}$ and let $A_i, B_i$ be $\mathcal{F}_i$-measurable random variables such that
\[
    A_i \leq \Delta_i \leq B_i
\]
holds almost surely for all $i \in \N$. Then,
\[
    \Pr{\sum_{i=1}^n \Delta_i \geq t} \leq \exp(\frac{-2t^2}{\sum_{i=1}^n \norm{B_i - A_i}_\infty^2}).
\]
\end{theorem}

In comparison to Hoeffding’s inequality, the Azuma-Hoeffding inequality allows for non-uniform boundedness and does not require independence of the random variables.

\begin{proof}
As in the proof of Hoeffding's inequality, we first apply the generic Chernoff bound to obtain
\begin{align*}
    \Pr{\sum_{i=1}^n \Delta_i \geq t} \leq \e^{-st} \, \Ex{\e^{s\sum_i \Delta_i}} &= \e^{-st} \, \Ex{\Ex{\e^{s\sum_i \Delta_i} \given \mathcal{F}_{n-1}}} \\
        &= \e^{-st} \, \Ex{\e^{s\sum_{i=1}^{n-1} \Delta_i} {\color{blue} \Ex{\e^{s\Delta_n} \given \mathcal{F}_{n-1}}}}.
\end{align*}
Next, applying Hoeffding's lemma to the inner expectation $\Ex{\e^{s\Delta_n} \given \mathcal{F}_{n-1}}$ yields
\[
    \Pr{\sum_{i=1}^n \Delta_i \geq t} \leq \e^{-st} \, \Ex{\e^{s\sum_{i=1}^{n-1} \Delta_i} {\color{blue} \e^{s^2\frac{(B_n - A_n)^2}{8}}}} \leq \e^{-st} \exp(\frac{s^2}{8}\norm{B_n - A_n}^2_\infty) \Ex{\e^{s\sum_{i=1}^{n-1} \Delta_i}}.
\]
Iteratively splitting off the $\Delta_i$ one-by-one and then applying Hoeffding's lemma, we eventually get
\[
    \Pr{\sum_{i=1}^n \Delta_i \geq t} \leq \e^{-st} \exp(\frac{s^2}{8}\sum_{i=1}^n \norm{B_i - A_i}_\infty^2).
\]
Optimizing the RHS over $s$ yields the desired result.
\end{proof}

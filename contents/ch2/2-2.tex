%% SECTION 2.2 %%
\section{Plug-In Decisions}
\label{subsec: plug-in decisions}

We know that the Bayes classifier $h^*(x) = \indSet{\eta(X) > \nicefrac{1}{2}}$ is our best guess of $Y$ based on the observation $X$. However, the function $\eta(x) = \Exp[Y \given X = x]$ is often unknown. Assume that we have a function $\tilde{\eta} \colon \mathcal{X} \to [0, 1]$ approximating $\eta$. In this case, it is natural to use the \emph{plug-in} classifier
\begin{equation}
\label{eq: plug-in classifier}
    \tilde{h}(x) = \begin{cases}
        1, \quad & \tilde{\eta}(x) > 1/2 \\
        0, \quad & \tilde{\eta}(x) \leq 1/2
    \end{cases}
\end{equation}
The error probability of this plug-in classifier is close to the Bayes error if $\tilde{\eta}$ is a good approximation of the regression function $\eta$.

\begin{theorem}
\label{thm: plug-in decision}
The excess risk $R(\tilde{h})$ of the plug-in classifier $\tilde{h}$ defined in \eqref{eq: plug-in classifier} satisfies
\[
    R(\tilde{h}) = \Exp[\abs{2\eta(X) - 1} \indSet{\tilde{h}(X) \neq h^*(X)}] \leq 2 \Exp[\abs{\eta(X) - \tilde{\eta}(X)}].
\]
\end{theorem}

\begin{proof}
The first identity is simply an application of Theorem \ref{thm: bayes classifier}. Thus, all we need to demonstrate is that the inequality
\[
    \abs{\eta(x) - 1/2} \leq \abs{\eta(x) - \tilde{\eta}(x)}
\]
holds on the set $A = \{ \tilde{h} \neq h^* \}$. First, assume that $\tilde{h}(x) = 0$ for $x \in A$. This implies that $h^*(x) = 1$. By definition of $\tilde{h}$ and $h^*$, we know that $\tilde{\eta}(x) \leq 1/2$ and $\eta(x) > 1/2$, and thus
\[
    \tilde{\eta}(x) \leq 1/2 < \eta(x).
\]
If $\tilde{h}(x) = 1$, then $h^*(x) = 0$, and consequently $\tilde{\eta}(x) > 1/2$ and $\eta(x) \leq 1/2$. Altogether, we have
\[
    \eta(x) \leq 1/2 < \tilde{\eta}(x).
\]
This proves the inequality stated at the beginning of this proof, and we're done.
\end{proof}

Since $\eta \colon \mathcal{X} \to [0, 1]$ takes values in the unit interval, the condition $\eta(x) > 1/2$ is equivalent to the inequality $\eta(x) > 1 - \eta(x)$. Assuming that we have two functions $\tilde{\eta}_1 \colon \mathcal{X} \to [0, 1]$ and $\tilde{\eta}_0 \colon \mathcal{X} \to [0, 1]$ such that $\tilde{\eta}_1$ approximates $\eta$ and $\tilde{\eta}_0$ approximates $1 - \eta$ with the constraint that $\tilde{\eta}_1$ and $\tilde{\eta}_0$ do \emph{not} sum to $1$, we can still build a plug-in classifier $\tilde{h}$ by plugging $\tilde{\eta}_1$ and $\tilde{\eta}_0$ into the condition $\eta(x) > 1 - \eta(x)$:
\begin{equation}
\label{eq: plug-in classifier not summing to 1}
    \tilde{h}(x) = \begin{cases}
        1, \quad & \tilde{\eta}_1(x) > \tilde{\eta}_0(x) \\
        0, \quad & \text{else}
    \end{cases}
\end{equation}
While this puts us into a slightly different situation compared to Theorem \ref{thm: plug-in decision}, a bound similar to that in Theorem \ref{thm: plug-in decision} still holds.

\begin{theorem}
The excess risk $R(\tilde{h})$ of the plug-in classifier $\tilde{h}$ defined in \eqref{eq: plug-in classifier not summing to 1} satisfies
\[
    R(\tilde{h}) \leq \Exp[\abs*{\eta(X) - \tilde{\eta}_1(X)}] + \Exp[\abs*{(1 - \eta(X)) - \tilde{\eta}_0(X)}].
\]
\end{theorem}

The proof is left to the reader. Finally, assume that the class-conditional densities $f_0$ and $f_1$ exist and are approximated by $\tilde{f}_0$ and $\tilde{f}_1$. Further, let the class probabilities $p = \P(Y = 1)$ and $1 - p = \P(Y = 0)$ be approximated by $\tilde{p}_1$ and $\tilde{p}_0$, respectively. In this case, one might use the plug-in decision
\begin{equation}
\label{eq: plug-in classifier for X with density}
    \tilde{h}(x) = \begin{cases}
        1, \quad & \tilde{f}_1(x) \tilde{p}_1 > \tilde{f}_0(x) \tilde{p}_0 \\
        0, \quad & \text{else}
    \end{cases}
\end{equation}
which is based on the Bayes classifier established in \eqref{eq: bayes classifier for X with density}.

\begin{proposition}
The excess risk $R(\tilde{h})$ of the plug-in classifier $\tilde{h}$ defined in \eqref{eq: plug-in classifier for X with density} satisfies
\[
    R(\tilde{h}) \leq \int_{\mathcal{X}} \abs*{pf_1(x) - \tilde{p}_1\tilde{f}_1(x) } \dx + \int_{\mathcal{X}} \abs*{(1 - p)f_0(x) - \tilde{p}_0\tilde{f}_0(x)} \dx.
\]
\end{proposition}

The proof is left to the reader once again.

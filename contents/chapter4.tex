%% CHAPTER 4 %%
\chapter{Vapnik-Chervonenkis (VC) Theory}
\label{ch: VC theory}

Once again, recall the general setup of Section \ref{subsec: empirical classification}: given a family $\mathcal{H}$ of classifiers, we can express the excess risk of the empirical risk minimizer $\hat h$ as
\[
    R(\hat h) = \underbrace{L(\hat h) - L(\bar h)}_{\substack{\text{estimation} \\ \text{error}}} + \underbrace{L(\bar h) - L^*}_{\substack{\text{approximation} \\ \text{error}}}
\]
where $\bar h \in \mathcal{H}$ is the oracle, i.e., the classifier in $\mathcal{H}$ that (somehow) minimizes the true risk over $\mathcal{H}$. The second term on the RHS is the approximation error, which remains fixed once we have settled on a family $\mathcal{H}$. Hence, we try to find bounds on the first term of the RHS, the estimation error. In Theorem \ref{thm: estimation error finite dictionary}, we had seen that the inequality
\[
    L(\hat h) - L(\bar h) < \sqrt{\frac{2 \log(2M / \delta)}{n}}
\]
holds with probability at least $1 - \delta$, if $\mathcal{H}$ is a \emph{finite} family of $M$ classifiers. Note that this upper bound for finite $\mathcal{H}$ cannot be applied to the case that $\mathcal{H}$ is \emph{infinite}. Essentially, to extend our previous results to the infinite case, we would need to show that only finitely many elements in a possibly infinite dictionary $\mathcal{H}$ \qq{really matter}. This exactly is the goal of the Vapnik-Chervonenkis (VC) theory developed in 1971.

We start with the somewhat unrealistic case that our family $\mathcal{H}$ of classifiers is (again) finite and that there exists a classifier $\bar h \in \mathcal{H}$ with zero error probability. In that case, the empirical risk minimizer $\hat h$ satisfies $\hat L_n(\hat h) = 0$ almost surely. Since $\hat h$ is the classifier in $\mathcal{H}$ minimizing the empirical risk, observe that
\[
    \Ex{\hat L_n(\hat h)} = \Ex{\min_{h \in \mathcal{H}} \hat L_n(h)} \leq \min_{h \in \mathcal{H}} \, \Ex{\hat L_n(h)} = \min_{h \in \mathcal{H}} L(h) = L(\bar h) = 0.
\]
Because $\hat L_n(\hat h)$ is non-negative, this implies that $\P(\hat L_n(\hat h) = 0) = 1$. While this tells us that the ERM almost surely has no \emph{empirical} error, we can also bound its \emph{true} error as follows:

\begin{theorem}[Vapnik and Chervonenkis, 1974]
Let $\card{\mathcal{H}} < \infty$ and $L(\bar h) = \min_{h \in \mathcal{H}} L(h) = 0$. Then, for every $n \in \N$, the empirical risk minimzer $\hat h$ satisfies
\[
    \P(L(\hat h) > \varepsilon) \leq \e^{-n \varepsilon} \, \card{\mathcal{H}}, \quad \varepsilon > 0
\]
and
\[
    \ex{L(\hat h)} \leq \frac{1 + \log(\card{\mathcal{H}})}{n}.
\]
\end{theorem}

\begin{proof}
Since $\hat L_n(\hat h) = 0$ almost surely, we have $\hat h \in \{h \in \mathcal{H} \with \hat L_n(h) = 0 \text{ a.s.}\}$, and hence
\begin{align*}
    \P(L(\hat h) > \varepsilon) &\leq \Pr{\max_{h \in \mathcal{H}, \, \hat L_n(h) = 0} L(h) > \varepsilon} = \Ex{\indSet{\max_{h \in \mathcal{H}, \, \hat L_n(h) = 0} L(h) > \varepsilon}} \\[4pt]
        &=\Ex{\max_{h\in\mathcal{H}} \, \mathbf{1}_{\{\hat L_n(h) = 0\}} \indSet{L(h) > \varepsilon}} \leq \sum_{h\in\mathcal{H}, \, L(h) > \varepsilon} \P(\hat L_n(h) = 0).
\end{align*}
Since the probability $\P(\hat L_n(h) = 0)$ that no $(X_i, Y_i)$ falls in the set $\set{(x, y) \with h(x) \neq y}$ is at most $(1 - \varepsilon)^n$ if the probability of this set\footnote{This is precisely the true loss $L(h) = \Pr{h(X) \neq Y}$, which is larger than $\varepsilon$ by assumption.} is larger than $\varepsilon$, we conclude
\[
    \P(L(\hat h_n) > \varepsilon) \leq \sum_{h\in\mathcal{H}, \, L(h) > \varepsilon} \P(\hat L_n(h) = 0) \leq (1 - \varepsilon)^n \, \card{\mathcal{H}} \leq \e^{-n \varepsilon} \, \card{\mathcal{H}}.
\]

To bound the expected error probability of $\hat h$ for $n \in \N$, first recall that the expected value of a non-negative random variable $Z \geq 0$ can be computed as follows:
\[
    \Ex{Z} = \int_0^\infty \Pr{X > t} \dt.
\]
Hence, we have
\begin{align*}
    \ex{L(\hat h)} &= \int_0^\infty \P(L(\hat h) > t) \dt \\
        &\leq u + \int_u^\infty \P(L(\hat h) > t) \dt \leq u + \card{\mathcal{H}} \int_u^\infty \e^{-nt} \dt = u + \frac{\e^{-nu}}{n} \, \card{\mathcal{H}}.
\end{align*}
Since $u > 0$ was arbitrary, we can minimize the RHS to obtain $u = \log(\card{\mathcal{H}}) / n$, giving the desired result.
\end{proof}

%% SECTION 4.4 %%
\section{The VC Inequality}
\label{sec: VC inequality}

We have now introduced all the necessary concepts to formulate the main result of this chapter: the VC inequality.

\begin{theorem}[VC Inequality]
Every family of sets $\mathcal{A}$ with VC dimension $VC(\mathcal{A}) = D$ satisfies
\[
    \Ex{ \sup_{A \in \mathcal{A}} \abs{\mu_n(A) - \mu(A)} } \leq 2 \sqrt{\frac{2D \log(2 \e n / D)}{n}}.
\]
\end{theorem}

We split the proof of this result into three steps:

\begin{enumerate}
    \item Using symmetrization, we prove that
    \[
        \Ex{ \sup_{A \in \mathcal{A}} \abs{\mu_n(A) - \mu(A)} } \leq 2 \mathcal{R}_n(\mathcal{A}).
    \]
    Note that this is precisely the bound \eqref{eq: symmetrization bound} obtained in Section \ref{sec: symmetrization}.

    \item We bound the Rademacher complexity of $\mathcal{A}$ using shatter coefficients:
    \[
        \mathcal{R}_n(\mathcal{A}) \leq \sqrt{ \frac{2 \log(2 \mathcal{S}_{\mathcal{A}}(n)) }{n} }.
    \]

    \item Finally, we bound the shatter coefficients $\mathcal{S}_{\mathcal{A}}(n)$ by the VC dimension $VC(\mathcal{A})$:
    \[
        \mathcal{S}_{\mathcal{A}}(n) \leq \left( \frac{\e n}{D} \right)^D, \quad D = \mathrm{VC}(\mathcal{A}).
    \]
    This is known as the \emph{Sauer-Shelah} lemma.
\end{enumerate}

Since the first step is already done, let's tackle step number 2. As an intermediate result, we prove a bound for the Rademacher complexity of finite subsets in $\R^n$. This will turn out to be immensely useful, as we can express the Rademacher complexity of a family of sets $\mathcal{A}$ in terms of the Rademacher complexity of finite subsets in $\R^n$, i.e., $\mathcal{R}_n(\mathcal{A}) = \sup_z \mathcal{R}_n(T(z))$, where $T(z)$ is defined as in \eqref{eq: T(z)}.

\begin{lemma}
\label{lem: bound on rademacher complexity of finite set}
For any finite set $B \subset \R^n$ (i.e., $\card{B} < \infty$) it holds
\[
    \mathcal{R}_n(B) \leq \max_{b \in B} \norm{b} \frac{\sqrt{2 \log(2 \, \card{B})}}{n},
\]
where $\norm{\cdot}$ denotes the Euclidean norm on $\R^d$.
\end{lemma}

\begin{proof}
By definition,
\[
    n \mathcal{R}_n(B) = \Ex{ \max_{b \in B} \abs{Z_b} },
\]
where $Z_b = \sum_{i=1}^n \sigma_i b_i$. Since $-\abs{b_i} \leq \sigma_i b_i \leq \abs{b_i}$ holds almost surely, Hoeffding's lemma (Lemma \ref{lem: hoeffding}) bounds the moment generating function of $Z_b$ by
\begin{equation}
\label{eq: proof of bound on rademacher complexity of finite set}
    \Ex{\exp(s Z_b)} = \prod_{i=1}^n \Ex{\exp(s \sigma_i b_i)} \leq \prod_{i=1}^n \exp(s^2 b_i^2 / 2) = \exp(s^2 \norm{b}^2 / 2).
\end{equation}
To bound the quantity $\Ex{\max_{b \in B} \abs{Z_b}}$ we are interested in, let $\overline{B} = B \cup -B$, where $-B = \set{-b \with b \in B}$. For $s>0$,
\[
    \Ex{\max_{b \in B} \abs{Z_b}} = \Ex{\max_{b \in \overline{B}} Z_b} = \frac{1}{s} \log( \exp( s \Ex{\max_{b \in \overline{B}} Z_b} )) \leq \frac{1}{s} \log(\Ex{ \exp( s \max_{b \in \overline{B}} Z_b )} ),
\]
where the last inequality follows from Jensen's inequality applied to the function $x \mapsto \exp(sx)$. We bound the maximum somewhat crudely by a sum to obtain
\begin{align*}
    \Ex{\max_{b \in B} \abs{Z_b}} &\leq \frac{1}{s} \log( \sum_{b \in \overline{B}} \Ex{\exp(s Z_b)} ) \\[4pt]
        &\leq \frac{\log(2 \, \card{B})}{s} + \frac{s}{2} \max_{b \in B} \norm{b}^2 = \frac{2 \log(2 \, \card{B}) + s^2 \max_{b \in B} \norm{b}^2}{2s},
\end{align*}
where the second inequality follows from \eqref{eq: proof of bound on rademacher complexity of finite set} and the observation that $\max_{b \in \overline{B}} \norm{b}^2 = \max_{b \in B} \norm{b}^2$ and $\card{\overline{B}} \leq 2 \, \card{B}$. Optimizing the RHS over $s$ yields the optimal solution
\[
    s^2 = \frac{2 \log(2 \, \card{B})}{\max_{b \in B} \norm{b}^2}.
\]
Plugging this back in yields the desired result, i.e.,
\[
    n \mathcal{R}_n(B) = \Ex{ \max_{b \in B} \abs{Z_b} } \leq \max_{b \in B} \norm{b} \sqrt{2 \log(2 \, \card{B})}.
\]
\end{proof}

By applying the previous result to the finite set $T(z)$, we obtain
\[
    \mathcal{R}_n(T(z)) \leq \max_{b \in T(z)} \norm{b} \frac{\sqrt{2 \log(2 \, \card{T(z)})}}{n}.
\]
Since each entry of a vector in $T(z)$ takes values in the set $\set{0, 1}$, we know that the norm $\norm{b}$ of any vector $b \in T(z)$ is at most $\sqrt{n}$. Further, we know that the shatter coefficients $\mathcal{S}_{\mathcal{A}}(n)$ of $\mathcal{A}$ depend (by definition) directly on the cardinality of the sets $T(z)$. Hence, the next result should come at no surprise.

\begin{proposition}
\label{prop: bound on rademacher complexity of family of sets}
For a family of sets $\mathcal{A}$, it holds
\[
    \mathcal{R}_n(\mathcal{A}) \leq \sqrt{ \frac{2 \log(2 \mathcal{S}_{\mathcal{A}}(n)) }{n} }.
\]
\end{proposition}

\begin{proof}
Observe that
\[
    \mathcal{R}_n(\mathcal{A}) = \sup_{z_1, \dots, z_n} \mathcal{R}_n(T(z)),
\]
by Lemma \ref{lem: rademacher complexity of family of sets}, where $T(z)$ is defined as in \eqref{eq: T(z)}. Since $T(z) \subset \set{0, 1}^n$, we have $\norm{b} \leq \sqrt{n}$ for all $b \in T(z)$ and all subsets $z = \set{z_1, \dots, z_n}$. Hence, by Lemma \ref{lem: bound on rademacher complexity of finite set}, we have
\[
    \mathcal{R}_n(\mathcal{A}) \leq \sup_{z_1, \dots, z_n} \sqrt{\frac{2 \log(2 \, \card{T(z)})}{n}}
\]
Finally, by the definition of the shatter coefficients of $\mathcal{A}$, we have
\[
    \card{T(z)} \leq \sup_{z_1, \dots, z_n} \card{T(z)} = \mathcal{S}_{\mathcal{A}}(n),
\]
and hence
\[
    \mathcal{R}_n(\mathcal{A}) \leq \sqrt{ \frac{2 \log(2 \mathcal{S}_{\mathcal{A}}(n)) }{n} }.
\]
\end{proof}

Up to this point, we have shown that
\begin{equation}
\label{eq: precursor to VC inequality}
    \Ex{ \sup_{A \in \mathcal{A}} \abs{\mu_n(A) - \mu(A)} } \leq 2 \mathcal{R}_n(\mathcal{A}) \leq 2 \sqrt{ \frac{2 \log(2 \mathcal{S}_{\mathcal{A}}(n)) }{n} }.
\end{equation}
Note that this bound would not be informative (in the sense that it does not imply convergence of the uniform deviations to zero as the sample size $n$ goes to infinity) if the shatter coefficients $\mathcal{S}_{\mathcal{A}}(n)$ were exponential in $n$. For example, if $\mathcal{S}_{\mathcal{A}}(n) = 2^n$ for $n \leq D$ and $\mathcal{S}_{\mathcal{A}}(n) = 2^n - 1$ for all $n > D$, the RHS of \eqref{eq: precursor to VC inequality} would be greater than $2$ for all $n$. The VC inequality suggests that this cannot be the case, and indeed, if the VC dimension of a family of sets is \emph{finite}, the shatter coefficients of $\mathcal{A}$ can be at most \emph{polynomial} in $n$. This result is known as the Sauer-Shelah lemma, which we now turn to.

\begin{lemma}[Sauer-Shelah]
For a family of sets $\mathcal{A}$ with finite VC dimension $VC(\mathcal{A}) = D$, the shatter coefficients satisfy
\[
    \mathcal{S}_{\mathcal{A}}(n) \leq \sum_{k=0}^D \binom{n}{k} \leq \left( \frac{\e n}{D} \right)^D, \quad n \in \N.
\]
\end{lemma}

For a proof of this result, we refer the reader to Theorem 13.2 of Section 13.1 in \cite[p.~216]{devroye1996probabilistic}. Replacing $\mathcal{S}_{\mathcal{A}}(n)$ with $(\e n/D)^D$ in \eqref{eq: precursor to VC inequality} clearly yields the VC inequality
\[
    \Ex{ \sup_{A \in \mathcal{A}} \abs{\mu_n(A) - \mu(A)} } \leq 2 \sqrt{\frac{2D \log(2 \e n / D)}{n}}.
\]

Recall that, using the bounded differences inequality (Theorem \ref{thm: bounded differences inequality}), we had shown that the bound
\[
    \sup_{A \in \mathcal{A}} \bigAbs{ \mu_n(A) - \mu(A) } < \Ex{\sup_{A \in \mathcal{A}} \bigAbs{ \mu_n(A) - \mu(A) }} + \sqrt{\frac{\log(\delta^{-1})}{2n}}
\]
holds with probability at least $1 - \delta$. Hence, we conclude:

\begin{corollary}[VC Inequality]
For a family of sets $\mathcal{A}$ with finite VC dimension $VC(\mathcal{A}) = D$, the upper bound
\[
    \sup_{A \in \mathcal{A}} \bigAbs{ \mu_n(A) - \mu(A) } < 2 \sqrt{\frac{2D \log(2 \e n / D)}{n}} + \sqrt{\frac{\log(\delta^{-1})}{2n}}
\]
holds with probability at least $1 - \delta$.
\end{corollary}

Note that, while this bound can be improved, it is always of rate $\sqrt{\log(n) / n}$, which is a \qq{slow rate}.

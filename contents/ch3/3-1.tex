%% SECTION 3.1 %%
\section{Azuma-Hoeffding Inequality}

To state the main result of this section, we first need to introduce the concept of \emph{martingales}.

\begin{definition}
Let $(\Omega, \mathcal{A}, \P)$ be a probability space.
\begin{enumerate}
    \item A \emph{filtration} $\set{\mathcal{F}_n}_{n \in \N}$ is an increasing sequence of sub-$\sigma$-algebras $\mathcal{F}_n$ of $\mathcal{A}$, i.e.,
    \[
        \mathcal{F}_1 \subset \mathcal{F}_2 \subset \dots \subset \mathcal{A}.
    \]

    \item A stochastic process $\set{X_n}_{n \in \N}$ is called a \emph{martingale} if, for every $n \in \N$,
    \begin{enumerate}
        \item $X_n$ is $\mathcal{F}_n$-measurable and integrable, i.e., $\Exp[\abs{X_n}] < \infty$,

        \item $\Exp[X_{n+1} \given \mathcal{F}_n] = X_n$ holds almost surely.
    \end{enumerate}
\end{enumerate}
\end{definition}

Loosely speaking, martingales are a generalization of sums of zero-mean, independent random variables: Let $\set{X_n}_{n \in \N}$ be a sequence of independent random variables with $\Exp[X_n] = 0$, and let $S_n = \sum_{i=1}^n X_i$. Then, for $\mathcal{F}_n = \sigma(X_1, \dots, X_n)$, we have
\[
    \Exp[S_{n+1} \given \mathcal{F}_n] = \Exp[X_{n+1} + S_n \given \mathcal{F}_n] = \Exp[X_{n+1}] + \Exp[S_n \given \mathcal{F}_n] = S_n,
\]
since $X_{n+1}$ is centered and independent of $\mathcal{F}_n$, and $S_n$ is $\mathcal{F}_n$-measurable.

Another definition we will need is that of a \emph{martingale difference sequence}.

\begin{definition}
Let $(\Omega, \mathcal{A}, \P)$ be a probability space with filtration $\set{\mathcal{F}_n}_{n \in \N}$. A stochastic process $\set{X_n}_{n \in \N}$ is called a \emph{martingale difference sequence (MDS)}, if
\begin{enumerate}
    \item $X_n$ is $\mathcal{F}_n$-measurable and integrable, i.e., $\Exp[\abs{X_n}] < \infty$,

    \item $\Exp[X_{n+1} \given \mathcal{F}_n] = 0$ holds almost surely.
\end{enumerate}
\end{definition}

Whenever we have a martingale $\set{X_n}_{n \in \N}$, we can construct a martingale difference sequence by setting $\Delta_n = X_n - X_{n-1}$, since then
\[
    \Exp[\Delta_{n+1} \given \mathcal{F}_n] = \Exp[X_{n+1} - X_n \given \mathcal{F}_n] = \Exp[X_{n+1} \given \mathcal{F}_n] - X_n = 0.
\]

With these definitions at hand, we can state the main result of this section:

\begin{theorem}[Azuma-Hoeffding]
\label{thm: azuma-hoeffding}
Assume that $\set{\Delta_i}_{i \in \N}$ is a martingale difference sequence with respect to a filtration $\set{\mathcal{F}_i}_{i \in \N}$ and let $A_i, B_i$ be $\mathcal{F}_i$-measurable random variables such that
\[
    A_i \leq \Delta_i \leq B_i
\]
holds almost surely for all $i \in \N$. Then,
\[
    \Prob\left(\sum_{i=1}^n \Delta_i \geq t\right) \leq \exp(\frac{-2t^2}{\sum_{i=1}^n \norm{B_i - A_i}_{\infty}^2}).
\]
\end{theorem}

In comparison to Hoeffding’s inequality, the Azuma-Hoeffding inequality allows for non-uniform boundedness and does \emph{not} require independence of the random variables.

\begin{proof}
For ease of notation, we write $S_k = \sum_{i=1}^k \Delta_i$, with $k = 1, \dots, n$. As in the proof of Hoeffding's inequality, we first apply the generic Chernoff bound to obtain
\begin{align*}
    \Prob(S_n \geq t) \leq \e^{-st} {\color{blue} \Exp[\exp(s S_n)]} &= \e^{-st} \Exp[\Exp[\exp(s S_n) \given \mathcal{F}_{n-1}]] \\
    &= \e^{-st} \Exp[\exp(s S_{n-1}) {\color{red} \Exp[\exp(s \Delta_n) \given \mathcal{F}_{n-1}]}].
\end{align*}
Next, applying Hoeffding's lemma (Lemma \ref{lem: hoeffding}) to the inner expectation ${\color{red} \Exp[\exp(s \Delta_n) \given \mathcal{F}_{n-1}]}$ yields
\[
    \Prob(S_n \geq t) \leq \e^{-st} \Exp\left[\exp(s S_{n-1}) \, {\color{red} \exp(s^2 \frac{(B_n - A_n)^2}{8})}\right] \leq \e^{-st} \exp(\frac{s^2}{8} \norm{B_n - A_n}_{\infty}^2) {\color{blue} \Exp[\exp(s S_{n-1})]}.
\]
Iteratively splitting off the $\Delta_i$ one-by-one and then applying Hoeffding's lemma, we eventually get
\[
    \Prob(S_n \geq t) \leq \e^{-st} \exp(\frac{s^2}{8} \sum_{i=1}^n \norm{B_i - A_i}_{\infty}^2).
\]
Optimizing the RHS over $s$ yields the desired result.
\end{proof}

%% SECTION 4.5 %%
\section{Application to the ERM}

Recall our goal that we started with at the beginning of this chapter: given a family $\mathcal{H}$ of classifiers, we wanted to bound the excess risk $R(\hat{h})$ of the empirical risk minimizer $\hat{h}$, which can be decomposed into estimation error and approximation error. Since the approximation error is fixed for a given family of classifiers $\mathcal{H}$, we have focused on bounding the estimation error $L(\hat{h}) - L(\bar{h})$. In Lemma \ref{lem: bound on estimation error}, we had already observed that the estimation error is bounded by
\[
    L(\hat{h}) - L(\bar{h}) \leq 2 \sup_{h \in \mathcal{H}} \abs*{\hat{L}_n(h) - L(h)}.
\]
Defining $A_h$ and $\mathcal{A}$ as in \eqref{eq: sets A_h} and \eqref{eq: collection of sets A_h}, respectively, and letting $\mu_n(A) = n^{-1} \sum_{i=1}^n \indSet{Z_i \in A}$ and $\mu(A) = \Prob(Z \in A)$, we had observed that $\sup_{h \in \mathcal{H}} \abs*{\hat{L}_n(h) - L(h)} = \sup_{A \in \mathcal{A}} \abs{\mu_n(A) - \mu(A)}$. Hence, the VC inequality tells us that
\begin{equation}
\label{eq: VC ineqaulity for ERM}
    \highlightMath{
        \Prob\left(\sup_{h \in \mathcal{H}} \abs*{\hat{L}_n(h) - L(h)} < 2 \sqrt{\frac{2 V_{\mathcal{A}} \log(2 \e n / V_{\mathcal{A}})}{n}} + \sqrt{\frac{\log(\delta^{-1})}{2n}}\right) \geq 1 - \delta,
    }
\end{equation}
where $V_{\mathcal{A}}$ denotes the VC dimension of $\mathcal{A} = \set{A_h \with h \in \mathcal{H}}$. However, the VC dimension of this class $\mathcal{A}$ is not very natural, and in many cases it is more convenient to consider the class
\begin{equation}
\label{eq: collection of sets bar A_h}
    \bar{\mathcal{A}} = \{\bar A_h \with h \in \mathcal{H}\}, \quad \bar A_h = \set{x \in \mathcal{X} \with h(x) = 1}.
\end{equation}
A priori it is not clear, how the VC dimension of the class $\mathcal{A}$ and the VC dimension of the class $\bar{\mathcal{A}}$ are related, if at all. However, as the next result shows, these two quantities actually coincide.

\begin{theorem}
Let $\mathcal{H}$ be a family of classifiers and let $\mathcal{A}$ and $\bar{\mathcal{A}}$ be defined as in \eqref{eq: collection of sets A_h} and \eqref{eq: collection of sets bar A_h}, respectively. Then,
\[
    \mathcal{S}_{\mathcal{A}}(n) = \mathcal{S}_{\bar{\mathcal{A}}}(n), \quad n \geq 1,
\]
which implies that $V_{\mathcal{A}} = V_{\bar{\mathcal{A}}}$.
\end{theorem}

\begin{proof}
Recall that the $n$-th shatter coefficent $\mathcal{S}_{\mathcal{A}}(n)$ is defined by $\mathcal{S}_{\mathcal{A}}(n) = \sup_{z \in \mathcal{Z}^n} \abs{T(z)}$, where $T(z)$ is the set of binary patterns that can be generated by the tuple $z = (z_1, \dots, z_n) \in \mathcal{Z}^n$, i.e.,
\[
    T(z) = \{(\indSet{z_1 \in A}, \dots, \indSet{z_n \in A})^{\top} \with A \in \mathcal{A}\}, \quad z_i = (x_i, y_i) \in \mathcal{X} \times \set{0, 1}.
\]
By definition of the collection of sets $\mathcal{A}$ in \eqref{eq: collection of sets A_h}, we can rewrite this as
\[
    T(z) = \{(\indSet{h(x_1) \neq y_1}, \dots, \indSet{h(x_n) \neq y_n})^{\top} \with h \in \mathcal{H}\}.
\]
Similarly, let $\bar{T}(z)$ denote the set of binary patterns generated by $z$ for the collection of sets $\bar{\mathcal{A}}$ defined in \eqref{eq: collection of sets bar A_h}, i.e.,
\[
    \bar{T}(z) = \{(\indSet{h(x_1) = 1}, \dots, \indSet{h(x_n) = 1})^{\top} \with h \in \mathcal{H}\}.
\]
Next, we fix $v \in \set{0, 1}$ and let $u \oplus v$ denote the logical XOR operation applied to $u \in \set{0, 1}$, i.e.,
\[
    \oplus \colon \set{0, 1} \to \set{0, 1}, \quad u \mapsto u \oplus v = \indSet{u \neq v}.
\]
The XOR operation is an involution, i.e., it satisfies $(u \oplus v) \oplus v = u$. In particular, the XOR operation is bijective. By applying the XOR operation to each entry of a vector, we have
\[
    (\indSet{h(x_1) \neq y_1}, \dots, \indSet{h(x_n) \neq y_n})^{\top} = (\indSet{h(x_1) = 1}, \dots, \indSet{h(x_n) = 1})^{\top} \oplus (y_1, \dots, y_n)^{\top}.
\]
Since the XOR operation is bijective, this tells us that the cardinalities of $T(z)$ and $\bar{T}(z)$ coincide. Consequently, so do the shatter coefficients and the VC dimension of the collections $\mathcal{A}$ and $\bar{\mathcal{A}}$.
\end{proof}

Based on our discussion of the emprical risk minimzer at the beginning of this section, we conclude the following:

\begin{corollary}
Let $\mathcal{H}$ be a family of classifiers such that the family $\bar{\mathcal{A}}$ defined in \eqref{eq: collection of sets bar A_h} has finite VC dimension $V_{\bar{\mathcal{A}}}$. Then, for $n > 2 V_{\mathcal{A}}$, the empirical risk minimizer $\hat{h}$ of $\mathcal{H}$ satisfies
\[
    L(\hat{h}) < L(\bar{h}) + 4 \sqrt{\frac{2 V_{\bar{\mathcal{A}}} \log(2 \e n / V_{\bar{\mathcal{A}}})}{n}} + \sqrt{\frac{2 \log(\delta^{-1})}{n}}
\]
with probability at least $1 - \delta$.
\end{corollary}
